\chapter*{Resumen}
\setlength{\parskip}{1ex}

La robótica ha experimentado un crecimiento exponencial en los últimos tiempos, lo que se traduce en una acentuada presencia en la vida cotidiana y profesional, siendo cada vez más y más importante especialistas formados en esta materia. Así, surgen recientemente entornos que acercan a los alumnos de distintas etapas la posibilidad de adquirir competencias en robótica, como JdeRobot-Academy, que pone a disposición de los estudiantes universitarios varios ejercicios clásicos de la robótica de forma sencilla, completa y eficaz.

Este proyecto se ha centrado en la elaboración de dos nuevas prácticas para JdeRobot-Academy. Para cada una de ellas se ha preparado una infraestructura (con robot real o simulado), un nodo académico, la comunicación que resuelve con robot y la interfaz gráfica, permitiendo a potenciales alumnos abstraerse de las complejidades involucradas y así centrarse en el algoritmo oportuno. También se han desarrollado sendas soluciones de referencia. 

La primera de las prácticas es \textit{``Follow Face''}, que permitirá al alumno trabajar con hardware real (cámara Sony Evi d100p) y adquirir conocimientos de procesado de imagen, empleando clasificadores en cascada para discernir si una imagen de entrada al sistema contiene caras o no, y en caso afirmativo las persiga usando un control visual reactivo.

La segunda práctica, \textit{``Laser Loc''}, busca resolver un problema de localización basada en láser, en el cual el estudiante deberá valerse del método de filtro de partículas para lograr que un robot estime su posición con errores tolerables en un mundo conocido, del cual se proporciona un mapa binario.