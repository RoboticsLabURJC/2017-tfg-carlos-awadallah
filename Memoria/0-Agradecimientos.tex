\newgeometry{top=1.6cm,left=2cm,right=2cm}
\chapter*{Agradecimientos}
\setlength{\parskip}{1ex}

Tenía muchas ganas de escribir esta parte del trabajo, ya que me brinda una oportunidad increíble para agradecer a todos aquellos que han estado a mi lado por el apoyo, el sustento (anímico y también económico) y la orientación que me han regalado sin esperar nada a cambio.

La primera persona en quien pienso es en mi madre, mi pilar de apoyo para seguir adelante y, sobre todo, mi referente en la vida. A ella le dedico este trabajo, espero que estas simples frases sirvan para hacerle ver que le estoy muy agradecido, y que no tengo palabras para expresar lo que siento hacia ella. También al resto de mi familia, mi hermana Shadia, mi tía Rosi y mi abuela MªRosa por sus continuos mensajes de apoyo, sin ellos no habría llegado donde estoy. También a Antonio, que siempre estuvo ahí, que fue la persona que me dio la oportunidad de estudiar lo que quería con su generosidad. A todos ellos, y también a todos mis demás familiares, gracias por todo.

Sigo con mi tutor José María, sin el cual nada de esto habría sido posible. Depositó su confianza en mí, aguantó pacientemente mis continuas preguntas e inquietudes y me orientó siempre que lo necesité para sacar el proyecto adelante. Gracias por tu dedicación, y también por transmitirme el gusto por la robótica durante el trabajo, las prácticas y las clases del grado, campo en el que probablemente enfocaré mi futuro profesional.

Gracias a mi pareja, Bea, que estuvo ahí absolutamente todos los días, que aunque nos separasen 2500 km de distancia siempre estuvo tan cerca como Diciembre de Enero. Por todos los mensajes, por todas las charlas de Skype y por todas y cada una de las palabras intercambiadas que, aunque nunca te lo dije, fueron vitales para seguir aquí día a día con ilusión. Muchas gracias por estos 4 años de apoyo, que se dice pronto.

Como olvidarme de mis amigos de siempre, en especial de Joel C., Joel Y. y Ricardo, y de mi amiga Lidia. Siempre supieron sacar lo mejor de mí, y nada cambió cuando pasamos a vernos sólo unas pocas veces al año. Por todos los momentos compartidos, que me han hecho llegar a ser quien soy.

Por último, pero no por ello menos importantes, mencionar a todas las personas que conocí en Madrid. A todos ustedes: gracias. Esto no habría sido fácil si no hubieran hecho más amena la vida cotidiana. A mis chicos y chicas de la universidad y a mis chicas del gimnasio, un fuerte abrazo.
\restoregeometry