\chapter{Conclusiones}\label{cap.conclusiones}
Tras detallar en profundidad las distintas prácticas e implementaciones de las mismas que componen este proyecto, este capítulo se ha reservado para ver hasta qué punto se han logrado los objetivos establecidos, para resumir los conocimientos adquiridos durante su desarrollo y para exponer las posibles mejoras que se pueden introducir en él que han salido a la luz durante la experimentación, así como trabajos futuros para completar o extender la funcionalidad.

\section{Conclusiones}
El objetivo global de este proyecto era crear una serie de prácticas docentes en el entorno JdeRobot-Academy para la instrucción acerca de técnicas, algoritmos y conceptos relacionas con el campo de la robótica, de manera que no sólo se adquieran conocimientos teóricos, sino que además estos puedan llevarse a la práctica. Este objetivo se ha alcanzado con éxito a través de las dos prácticas totalmente funcionales y testadas convenientemente a través de \textit{beta-testers} para comprobar su correcto funcionamiento. No obstante, dicho objetivo principal se subdivide en distintos objetivos secundarios, de los cuales a continuación se comentará hasta qué punto se ha logrado y cómo se ha logrado. 

El primero de ellos consistía en la creación de la práctica \textit{Follow face}, la cual  involucraba hardware real. El objetivo se cumplió satisfactoriamente con el desarrollo de la infraestructura y el nodo académico, además de con los \textit{drivers} que permitieron la comunicación entre el nodo y el robot. Hubo que lidiar con ciertos parámetros implícitos en el robot real que muchas veces no son tenidos en cuenta en simulaciones, como la presencia de un \textit{buffer} de mensajes y con el tiempo físico de ejecución de los movimientos del cuello mecánico. Se ha dejado resuelta toda la configuración y se ha añadido un fichero de instrucciones y enunciado que ayudará a futuros usuarios, teniendo así una práctica totalmente funcional y orientada.

Otro de los subobjetivos perseguidos fue la elaboración de una segunda práctica: \textit{Laser Loc}. Aunque esta implicaba bastante complejidad gráfica y mucho peso computacional, el objetivo se alcanzó utilizando técnicas de programación multihilo, precomputación y un manejo inteligente de los datos involucrados. En este caso, se preparó un mundo y un modelo de simulación (tanto en aspecto como en funcionalidad), se adaptó un nodo académico y su infraestructura, y se dejó preparado para incrustar una solución. 

El siguiente subojbetivo tenía que ver con los dos anteriores, y se trataba de construir dos soluciones de referencia (una para cada una de las prácticas), lo cual no sólo serviría como modelo de posibles soluciones para el ejercicio, sino que también traería consigo el estudio y asimilación de técnicas de control, procesado, captación y actuación que emplean los robots reales. 

\begin{itemize}
	\item[--] En el caso de la primera práctica, se hizo uso del aprendizaje máquina para entrenar un sistema básico capaz de reconocer caras en posición frontal y otro adicional para reconocer ojos, que juntos permitieron discriminar si una imagen dada contenía caras o no. Estos datos analizados sirvieron posteriormente para implementar un algoritmo de control gradual, de tal manera que la segmentación de la imagen se traducía en órdenes para los motores de la cámara Sony Evi d100p, la cual comenzaba a seguir al individuo de la imagen. Para hacer la solución algo más completa, también se incluyó una lógica de gestión en caso de no detectar rostros, lo cual hacía que el tanto el algoritmo como el hardware no quedasen ociosos.
	\item[--] Para la práctica de localización se propuso un algoritmo basado en la serie de métodos secuenciales  de Montecarlo que utilizara el filtro de partículas. Este proceso involucró distintos algoritmos que solventaban distintas partes del mismo, como pueden ser el algoritmo de la ruleta para la selección de una partícula a evolucionar, la geometría euclidiana para materializar desplazamientos en las partículas, o la estadística básica para implementar una función de salud o modelo de observación de partículas. Esta solución se completó con técnicas de mejora de la eficiencia, con soporte para la localización en movimiento y con una serie de ayudas gráficas.
\end{itemize}

Ambas permitieron alcanzar este tercer objetivo.

En cuanto al cuarto propósito, queríamos estudiar los requisitos necesarios para construir prácticas que estuviesen disponibles para la mayoría de usuarios. Esto ha sido posible gracias a la plataforma Jupyter, que nos permitió migrar nuestras aplicaciones de escritorio a una versión web, que conservaba la funcionalidad de las originales. Esto,  junto con las comunicaciones a través del \textit{middleware} ROS, hizo que nuestras prácticas redujeran en gran medida el número de componentes a instalar, además de abrir la puerta para su disponibilidad en distintas plataformas.

Por último, y siendo quizás el objetivo más importante dentro de la subdivisión, se perseguía adquirir el conocimiento y las habilidades necesarias para abordar un problema real de ingeniería, en el cual hubieran involucrados componentes hardware y software, y sobre todo en el que se partiese del principio, sin funcionalidad ni módulos resueltos. Así, se ha logrado integrar los conceptos que sustentan la infraestructura que hay bajo un proyecto de robótica, las interfaces involucradas, los componentes y, finalmente, la parte visible por el usuario. Además, por el camino se han adquirido otra serie de competencias útiles y presentes en muchos proyectos, como el modelado de elementos de simulación, el diseño y manejo de interfaces gráficas, el tratamiento de imagen y distintas técnicas de programación. Todo ello hará que en el futuro sea más fácil abordar proyectos de esta envergadura, gracias al haber realizado distintas pruebas y experimentos, cambios de la técnica inicial empleada o modificaciones del método de trabajo hasta alcanzar los objetivos planteados.

\section{Líneas de mejora}
Debido al tiempo del que se ha dispuesto para preparar el proyecto (finito), hemos pensado en una serie de posibles mejoras sobre lo realizado que listaremos a continuación:

\begin{itemize}
	\item La solución propuesta para la práctica Follow Face es capaz de reconocer caras frontales, y en el mejor de los casos rostros girados como mucho 45º desde el eje frontal. Una posible mejora podría ser entrenar el sistema de clasificación con observaciones que incluyan caras rotadas con distintos ángulos, a fin de reconocer todos los casos positivos.
	\item Por otro lado, las comunicaciones en esa práctica emplean dos drivers, uno de los cuales utiliza interfaces ICE. Para la mejora del atributo de accesibilidad a la práctica se podría tratar de crear uno equivalente que se comunicase a través de mensajes de ROS.
	\item Ya en cuanto a la segunda de las prácticas, el elevado peso computacional hace que, incluso con las técnicas de optimización empleadas, la eficiencia siga siendo un punto que se puede pulir. Se habría de estudiar nuevas modos de optimización y su compatibilidad con el nodo.
	\item Además, un modelado meticuloso del espacio de simulación permitiría obtener errores casi despreciables. Se podría estudiar herramientas de modelado para componer un mapa que se adapte perfectamente al entorno simulado.
\end{itemize}

\section{Trabajos futuros}
El desarrollo de este trabajo ha abierto las puertas a distintas ideas que pueden estudiarse y realizarse en próximos proyectos:

El más destacado es el de alcanzar enteramente el objetivo de unas aplicaciones multiplataforma. El entorno JdeRobot ha desarrollado recientemente una herramienta que permite utilizar el cliente web de Gazebo (\textit{Gzweb}) en consonancia con la plataforma Jupyter para habilitar un servidor que contenga las prácticas, ejecutables en cualquier lugar y a través de cualquier plataforma. Aunque aún está en proceso de depuración, más adelante resultaría interesante retocar las prácticas creadas para que estén disponibles a través de dicha herramienta. 

Otro tema interesante de estudio relacionando con la segunda práctica es el de la localización basada en visión. Se podría construir una nueva práctica, basada en el nodo académico ya creado (o una versión muy parecida) para abordar el problema de la localización desde otro tipo de sensores, como una o varias cámaras. Incluso, podría enriquecerse el modelo empleado para incorporar distintos sensores de diferentes tipos que recogieran información que compusiera unas solución mucho más robusta.

Así mismo, con la primera práctica resultaría interesante extender el componente hardware: montar la cámara en un pequeño robot móvil, y abordar el seguimiento de caras en el espacio 3D. Para ello, sería necesario explorar los tipos de robots móviles, sus motores y sus formas de comunicación, para luego hacerlos compatibles con el nodo e implementar una lógica de seguimiento de personas.

Por último, siempre quedan abiertos muchos otros frentes para abordar las mismas prácticas haciendo uso de otro tipo de algoritmos, para así poder realizar una comparación entre ellos y extraer conclusiones acerca de la validez, ventajas y desventajas de cada uno. 